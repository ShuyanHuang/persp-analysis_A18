
% Default to the notebook output style

    


% Inherit from the specified cell style.




    
\documentclass[11pt]{article}

    
    
    \usepackage[T1]{fontenc}
    % Nicer default font (+ math font) than Computer Modern for most use cases
    \usepackage{mathpazo}

    % Basic figure setup, for now with no caption control since it's done
    % automatically by Pandoc (which extracts ![](path) syntax from Markdown).
    \usepackage{graphicx}
    % We will generate all images so they have a width \maxwidth. This means
    % that they will get their normal width if they fit onto the page, but
    % are scaled down if they would overflow the margins.
    \makeatletter
    \def\maxwidth{\ifdim\Gin@nat@width>\linewidth\linewidth
    \else\Gin@nat@width\fi}
    \makeatother
    \let\Oldincludegraphics\includegraphics
    % Set max figure width to be 80% of text width, for now hardcoded.
    \renewcommand{\includegraphics}[1]{\Oldincludegraphics[width=.8\maxwidth]{#1}}
    % Ensure that by default, figures have no caption (until we provide a
    % proper Figure object with a Caption API and a way to capture that
    % in the conversion process - todo).
    \usepackage{caption}
    \DeclareCaptionLabelFormat{nolabel}{}
    \captionsetup{labelformat=nolabel}

    \usepackage{adjustbox} % Used to constrain images to a maximum size 
    \usepackage{xcolor} % Allow colors to be defined
    \usepackage{enumerate} % Needed for markdown enumerations to work
    \usepackage{geometry} % Used to adjust the document margins
    \usepackage{amsmath} % Equations
    \usepackage{amssymb} % Equations
    \usepackage{textcomp} % defines textquotesingle
    % Hack from http://tex.stackexchange.com/a/47451/13684:
    \AtBeginDocument{%
        \def\PYZsq{\textquotesingle}% Upright quotes in Pygmentized code
    }
    \usepackage{upquote} % Upright quotes for verbatim code
    \usepackage{eurosym} % defines \euro
    \usepackage[mathletters]{ucs} % Extended unicode (utf-8) support
    \usepackage[utf8x]{inputenc} % Allow utf-8 characters in the tex document
    \usepackage{fancyvrb} % verbatim replacement that allows latex
    \usepackage{grffile} % extends the file name processing of package graphics 
                         % to support a larger range 
    % The hyperref package gives us a pdf with properly built
    % internal navigation ('pdf bookmarks' for the table of contents,
    % internal cross-reference links, web links for URLs, etc.)
    \usepackage{hyperref}
    \usepackage{longtable} % longtable support required by pandoc >1.10
    \usepackage{booktabs}  % table support for pandoc > 1.12.2
    \usepackage[inline]{enumitem} % IRkernel/repr support (it uses the enumerate* environment)
    \usepackage[normalem]{ulem} % ulem is needed to support strikethroughs (\sout)
                                % normalem makes italics be italics, not underlines
    

    
    
    % Colors for the hyperref package
    \definecolor{urlcolor}{rgb}{0,.145,.698}
    \definecolor{linkcolor}{rgb}{.71,0.21,0.01}
    \definecolor{citecolor}{rgb}{.12,.54,.11}

    % ANSI colors
    \definecolor{ansi-black}{HTML}{3E424D}
    \definecolor{ansi-black-intense}{HTML}{282C36}
    \definecolor{ansi-red}{HTML}{E75C58}
    \definecolor{ansi-red-intense}{HTML}{B22B31}
    \definecolor{ansi-green}{HTML}{00A250}
    \definecolor{ansi-green-intense}{HTML}{007427}
    \definecolor{ansi-yellow}{HTML}{DDB62B}
    \definecolor{ansi-yellow-intense}{HTML}{B27D12}
    \definecolor{ansi-blue}{HTML}{208FFB}
    \definecolor{ansi-blue-intense}{HTML}{0065CA}
    \definecolor{ansi-magenta}{HTML}{D160C4}
    \definecolor{ansi-magenta-intense}{HTML}{A03196}
    \definecolor{ansi-cyan}{HTML}{60C6C8}
    \definecolor{ansi-cyan-intense}{HTML}{258F8F}
    \definecolor{ansi-white}{HTML}{C5C1B4}
    \definecolor{ansi-white-intense}{HTML}{A1A6B2}

    % commands and environments needed by pandoc snippets
    % extracted from the output of `pandoc -s`
    \providecommand{\tightlist}{%
      \setlength{\itemsep}{0pt}\setlength{\parskip}{0pt}}
    \DefineVerbatimEnvironment{Highlighting}{Verbatim}{commandchars=\\\{\}}
    % Add ',fontsize=\small' for more characters per line
    \newenvironment{Shaded}{}{}
    \newcommand{\KeywordTok}[1]{\textcolor[rgb]{0.00,0.44,0.13}{\textbf{{#1}}}}
    \newcommand{\DataTypeTok}[1]{\textcolor[rgb]{0.56,0.13,0.00}{{#1}}}
    \newcommand{\DecValTok}[1]{\textcolor[rgb]{0.25,0.63,0.44}{{#1}}}
    \newcommand{\BaseNTok}[1]{\textcolor[rgb]{0.25,0.63,0.44}{{#1}}}
    \newcommand{\FloatTok}[1]{\textcolor[rgb]{0.25,0.63,0.44}{{#1}}}
    \newcommand{\CharTok}[1]{\textcolor[rgb]{0.25,0.44,0.63}{{#1}}}
    \newcommand{\StringTok}[1]{\textcolor[rgb]{0.25,0.44,0.63}{{#1}}}
    \newcommand{\CommentTok}[1]{\textcolor[rgb]{0.38,0.63,0.69}{\textit{{#1}}}}
    \newcommand{\OtherTok}[1]{\textcolor[rgb]{0.00,0.44,0.13}{{#1}}}
    \newcommand{\AlertTok}[1]{\textcolor[rgb]{1.00,0.00,0.00}{\textbf{{#1}}}}
    \newcommand{\FunctionTok}[1]{\textcolor[rgb]{0.02,0.16,0.49}{{#1}}}
    \newcommand{\RegionMarkerTok}[1]{{#1}}
    \newcommand{\ErrorTok}[1]{\textcolor[rgb]{1.00,0.00,0.00}{\textbf{{#1}}}}
    \newcommand{\NormalTok}[1]{{#1}}
    
    % Additional commands for more recent versions of Pandoc
    \newcommand{\ConstantTok}[1]{\textcolor[rgb]{0.53,0.00,0.00}{{#1}}}
    \newcommand{\SpecialCharTok}[1]{\textcolor[rgb]{0.25,0.44,0.63}{{#1}}}
    \newcommand{\VerbatimStringTok}[1]{\textcolor[rgb]{0.25,0.44,0.63}{{#1}}}
    \newcommand{\SpecialStringTok}[1]{\textcolor[rgb]{0.73,0.40,0.53}{{#1}}}
    \newcommand{\ImportTok}[1]{{#1}}
    \newcommand{\DocumentationTok}[1]{\textcolor[rgb]{0.73,0.13,0.13}{\textit{{#1}}}}
    \newcommand{\AnnotationTok}[1]{\textcolor[rgb]{0.38,0.63,0.69}{\textbf{\textit{{#1}}}}}
    \newcommand{\CommentVarTok}[1]{\textcolor[rgb]{0.38,0.63,0.69}{\textbf{\textit{{#1}}}}}
    \newcommand{\VariableTok}[1]{\textcolor[rgb]{0.10,0.09,0.49}{{#1}}}
    \newcommand{\ControlFlowTok}[1]{\textcolor[rgb]{0.00,0.44,0.13}{\textbf{{#1}}}}
    \newcommand{\OperatorTok}[1]{\textcolor[rgb]{0.40,0.40,0.40}{{#1}}}
    \newcommand{\BuiltInTok}[1]{{#1}}
    \newcommand{\ExtensionTok}[1]{{#1}}
    \newcommand{\PreprocessorTok}[1]{\textcolor[rgb]{0.74,0.48,0.00}{{#1}}}
    \newcommand{\AttributeTok}[1]{\textcolor[rgb]{0.49,0.56,0.16}{{#1}}}
    \newcommand{\InformationTok}[1]{\textcolor[rgb]{0.38,0.63,0.69}{\textbf{\textit{{#1}}}}}
    \newcommand{\WarningTok}[1]{\textcolor[rgb]{0.38,0.63,0.69}{\textbf{\textit{{#1}}}}}
    
    
    % Define a nice break command that doesn't care if a line doesn't already
    % exist.
    \def\br{\hspace*{\fill} \\* }
    % Math Jax compatability definitions
    \def\gt{>}
    \def\lt{<}
    % Document parameters
    \title{assign2}
    
    
    

    % Pygments definitions
    
\makeatletter
\def\PY@reset{\let\PY@it=\relax \let\PY@bf=\relax%
    \let\PY@ul=\relax \let\PY@tc=\relax%
    \let\PY@bc=\relax \let\PY@ff=\relax}
\def\PY@tok#1{\csname PY@tok@#1\endcsname}
\def\PY@toks#1+{\ifx\relax#1\empty\else%
    \PY@tok{#1}\expandafter\PY@toks\fi}
\def\PY@do#1{\PY@bc{\PY@tc{\PY@ul{%
    \PY@it{\PY@bf{\PY@ff{#1}}}}}}}
\def\PY#1#2{\PY@reset\PY@toks#1+\relax+\PY@do{#2}}

\expandafter\def\csname PY@tok@w\endcsname{\def\PY@tc##1{\textcolor[rgb]{0.73,0.73,0.73}{##1}}}
\expandafter\def\csname PY@tok@c\endcsname{\let\PY@it=\textit\def\PY@tc##1{\textcolor[rgb]{0.25,0.50,0.50}{##1}}}
\expandafter\def\csname PY@tok@cp\endcsname{\def\PY@tc##1{\textcolor[rgb]{0.74,0.48,0.00}{##1}}}
\expandafter\def\csname PY@tok@k\endcsname{\let\PY@bf=\textbf\def\PY@tc##1{\textcolor[rgb]{0.00,0.50,0.00}{##1}}}
\expandafter\def\csname PY@tok@kp\endcsname{\def\PY@tc##1{\textcolor[rgb]{0.00,0.50,0.00}{##1}}}
\expandafter\def\csname PY@tok@kt\endcsname{\def\PY@tc##1{\textcolor[rgb]{0.69,0.00,0.25}{##1}}}
\expandafter\def\csname PY@tok@o\endcsname{\def\PY@tc##1{\textcolor[rgb]{0.40,0.40,0.40}{##1}}}
\expandafter\def\csname PY@tok@ow\endcsname{\let\PY@bf=\textbf\def\PY@tc##1{\textcolor[rgb]{0.67,0.13,1.00}{##1}}}
\expandafter\def\csname PY@tok@nb\endcsname{\def\PY@tc##1{\textcolor[rgb]{0.00,0.50,0.00}{##1}}}
\expandafter\def\csname PY@tok@nf\endcsname{\def\PY@tc##1{\textcolor[rgb]{0.00,0.00,1.00}{##1}}}
\expandafter\def\csname PY@tok@nc\endcsname{\let\PY@bf=\textbf\def\PY@tc##1{\textcolor[rgb]{0.00,0.00,1.00}{##1}}}
\expandafter\def\csname PY@tok@nn\endcsname{\let\PY@bf=\textbf\def\PY@tc##1{\textcolor[rgb]{0.00,0.00,1.00}{##1}}}
\expandafter\def\csname PY@tok@ne\endcsname{\let\PY@bf=\textbf\def\PY@tc##1{\textcolor[rgb]{0.82,0.25,0.23}{##1}}}
\expandafter\def\csname PY@tok@nv\endcsname{\def\PY@tc##1{\textcolor[rgb]{0.10,0.09,0.49}{##1}}}
\expandafter\def\csname PY@tok@no\endcsname{\def\PY@tc##1{\textcolor[rgb]{0.53,0.00,0.00}{##1}}}
\expandafter\def\csname PY@tok@nl\endcsname{\def\PY@tc##1{\textcolor[rgb]{0.63,0.63,0.00}{##1}}}
\expandafter\def\csname PY@tok@ni\endcsname{\let\PY@bf=\textbf\def\PY@tc##1{\textcolor[rgb]{0.60,0.60,0.60}{##1}}}
\expandafter\def\csname PY@tok@na\endcsname{\def\PY@tc##1{\textcolor[rgb]{0.49,0.56,0.16}{##1}}}
\expandafter\def\csname PY@tok@nt\endcsname{\let\PY@bf=\textbf\def\PY@tc##1{\textcolor[rgb]{0.00,0.50,0.00}{##1}}}
\expandafter\def\csname PY@tok@nd\endcsname{\def\PY@tc##1{\textcolor[rgb]{0.67,0.13,1.00}{##1}}}
\expandafter\def\csname PY@tok@s\endcsname{\def\PY@tc##1{\textcolor[rgb]{0.73,0.13,0.13}{##1}}}
\expandafter\def\csname PY@tok@sd\endcsname{\let\PY@it=\textit\def\PY@tc##1{\textcolor[rgb]{0.73,0.13,0.13}{##1}}}
\expandafter\def\csname PY@tok@si\endcsname{\let\PY@bf=\textbf\def\PY@tc##1{\textcolor[rgb]{0.73,0.40,0.53}{##1}}}
\expandafter\def\csname PY@tok@se\endcsname{\let\PY@bf=\textbf\def\PY@tc##1{\textcolor[rgb]{0.73,0.40,0.13}{##1}}}
\expandafter\def\csname PY@tok@sr\endcsname{\def\PY@tc##1{\textcolor[rgb]{0.73,0.40,0.53}{##1}}}
\expandafter\def\csname PY@tok@ss\endcsname{\def\PY@tc##1{\textcolor[rgb]{0.10,0.09,0.49}{##1}}}
\expandafter\def\csname PY@tok@sx\endcsname{\def\PY@tc##1{\textcolor[rgb]{0.00,0.50,0.00}{##1}}}
\expandafter\def\csname PY@tok@m\endcsname{\def\PY@tc##1{\textcolor[rgb]{0.40,0.40,0.40}{##1}}}
\expandafter\def\csname PY@tok@gh\endcsname{\let\PY@bf=\textbf\def\PY@tc##1{\textcolor[rgb]{0.00,0.00,0.50}{##1}}}
\expandafter\def\csname PY@tok@gu\endcsname{\let\PY@bf=\textbf\def\PY@tc##1{\textcolor[rgb]{0.50,0.00,0.50}{##1}}}
\expandafter\def\csname PY@tok@gd\endcsname{\def\PY@tc##1{\textcolor[rgb]{0.63,0.00,0.00}{##1}}}
\expandafter\def\csname PY@tok@gi\endcsname{\def\PY@tc##1{\textcolor[rgb]{0.00,0.63,0.00}{##1}}}
\expandafter\def\csname PY@tok@gr\endcsname{\def\PY@tc##1{\textcolor[rgb]{1.00,0.00,0.00}{##1}}}
\expandafter\def\csname PY@tok@ge\endcsname{\let\PY@it=\textit}
\expandafter\def\csname PY@tok@gs\endcsname{\let\PY@bf=\textbf}
\expandafter\def\csname PY@tok@gp\endcsname{\let\PY@bf=\textbf\def\PY@tc##1{\textcolor[rgb]{0.00,0.00,0.50}{##1}}}
\expandafter\def\csname PY@tok@go\endcsname{\def\PY@tc##1{\textcolor[rgb]{0.53,0.53,0.53}{##1}}}
\expandafter\def\csname PY@tok@gt\endcsname{\def\PY@tc##1{\textcolor[rgb]{0.00,0.27,0.87}{##1}}}
\expandafter\def\csname PY@tok@err\endcsname{\def\PY@bc##1{\setlength{\fboxsep}{0pt}\fcolorbox[rgb]{1.00,0.00,0.00}{1,1,1}{\strut ##1}}}
\expandafter\def\csname PY@tok@kc\endcsname{\let\PY@bf=\textbf\def\PY@tc##1{\textcolor[rgb]{0.00,0.50,0.00}{##1}}}
\expandafter\def\csname PY@tok@kd\endcsname{\let\PY@bf=\textbf\def\PY@tc##1{\textcolor[rgb]{0.00,0.50,0.00}{##1}}}
\expandafter\def\csname PY@tok@kn\endcsname{\let\PY@bf=\textbf\def\PY@tc##1{\textcolor[rgb]{0.00,0.50,0.00}{##1}}}
\expandafter\def\csname PY@tok@kr\endcsname{\let\PY@bf=\textbf\def\PY@tc##1{\textcolor[rgb]{0.00,0.50,0.00}{##1}}}
\expandafter\def\csname PY@tok@bp\endcsname{\def\PY@tc##1{\textcolor[rgb]{0.00,0.50,0.00}{##1}}}
\expandafter\def\csname PY@tok@fm\endcsname{\def\PY@tc##1{\textcolor[rgb]{0.00,0.00,1.00}{##1}}}
\expandafter\def\csname PY@tok@vc\endcsname{\def\PY@tc##1{\textcolor[rgb]{0.10,0.09,0.49}{##1}}}
\expandafter\def\csname PY@tok@vg\endcsname{\def\PY@tc##1{\textcolor[rgb]{0.10,0.09,0.49}{##1}}}
\expandafter\def\csname PY@tok@vi\endcsname{\def\PY@tc##1{\textcolor[rgb]{0.10,0.09,0.49}{##1}}}
\expandafter\def\csname PY@tok@vm\endcsname{\def\PY@tc##1{\textcolor[rgb]{0.10,0.09,0.49}{##1}}}
\expandafter\def\csname PY@tok@sa\endcsname{\def\PY@tc##1{\textcolor[rgb]{0.73,0.13,0.13}{##1}}}
\expandafter\def\csname PY@tok@sb\endcsname{\def\PY@tc##1{\textcolor[rgb]{0.73,0.13,0.13}{##1}}}
\expandafter\def\csname PY@tok@sc\endcsname{\def\PY@tc##1{\textcolor[rgb]{0.73,0.13,0.13}{##1}}}
\expandafter\def\csname PY@tok@dl\endcsname{\def\PY@tc##1{\textcolor[rgb]{0.73,0.13,0.13}{##1}}}
\expandafter\def\csname PY@tok@s2\endcsname{\def\PY@tc##1{\textcolor[rgb]{0.73,0.13,0.13}{##1}}}
\expandafter\def\csname PY@tok@sh\endcsname{\def\PY@tc##1{\textcolor[rgb]{0.73,0.13,0.13}{##1}}}
\expandafter\def\csname PY@tok@s1\endcsname{\def\PY@tc##1{\textcolor[rgb]{0.73,0.13,0.13}{##1}}}
\expandafter\def\csname PY@tok@mb\endcsname{\def\PY@tc##1{\textcolor[rgb]{0.40,0.40,0.40}{##1}}}
\expandafter\def\csname PY@tok@mf\endcsname{\def\PY@tc##1{\textcolor[rgb]{0.40,0.40,0.40}{##1}}}
\expandafter\def\csname PY@tok@mh\endcsname{\def\PY@tc##1{\textcolor[rgb]{0.40,0.40,0.40}{##1}}}
\expandafter\def\csname PY@tok@mi\endcsname{\def\PY@tc##1{\textcolor[rgb]{0.40,0.40,0.40}{##1}}}
\expandafter\def\csname PY@tok@il\endcsname{\def\PY@tc##1{\textcolor[rgb]{0.40,0.40,0.40}{##1}}}
\expandafter\def\csname PY@tok@mo\endcsname{\def\PY@tc##1{\textcolor[rgb]{0.40,0.40,0.40}{##1}}}
\expandafter\def\csname PY@tok@ch\endcsname{\let\PY@it=\textit\def\PY@tc##1{\textcolor[rgb]{0.25,0.50,0.50}{##1}}}
\expandafter\def\csname PY@tok@cm\endcsname{\let\PY@it=\textit\def\PY@tc##1{\textcolor[rgb]{0.25,0.50,0.50}{##1}}}
\expandafter\def\csname PY@tok@cpf\endcsname{\let\PY@it=\textit\def\PY@tc##1{\textcolor[rgb]{0.25,0.50,0.50}{##1}}}
\expandafter\def\csname PY@tok@c1\endcsname{\let\PY@it=\textit\def\PY@tc##1{\textcolor[rgb]{0.25,0.50,0.50}{##1}}}
\expandafter\def\csname PY@tok@cs\endcsname{\let\PY@it=\textit\def\PY@tc##1{\textcolor[rgb]{0.25,0.50,0.50}{##1}}}

\def\PYZbs{\char`\\}
\def\PYZus{\char`\_}
\def\PYZob{\char`\{}
\def\PYZcb{\char`\}}
\def\PYZca{\char`\^}
\def\PYZam{\char`\&}
\def\PYZlt{\char`\<}
\def\PYZgt{\char`\>}
\def\PYZsh{\char`\#}
\def\PYZpc{\char`\%}
\def\PYZdl{\char`\$}
\def\PYZhy{\char`\-}
\def\PYZsq{\char`\'}
\def\PYZdq{\char`\"}
\def\PYZti{\char`\~}
% for compatibility with earlier versions
\def\PYZat{@}
\def\PYZlb{[}
\def\PYZrb{]}
\makeatother


    % Exact colors from NB
    \definecolor{incolor}{rgb}{0.0, 0.0, 0.5}
    \definecolor{outcolor}{rgb}{0.545, 0.0, 0.0}



    
    % Prevent overflowing lines due to hard-to-break entities
    \sloppy 
    % Setup hyperref package
    \hypersetup{
      breaklinks=true,  % so long urls are correctly broken across lines
      colorlinks=true,
      urlcolor=urlcolor,
      linkcolor=linkcolor,
      citecolor=citecolor,
      }
    % Slightly bigger margins than the latex defaults
    
    \geometry{verbose,tmargin=1in,bmargin=1in,lmargin=1in,rmargin=1in}
    
    

    \begin{document}
    
    
    \maketitle
    
    

    
    by Shuyan Huang

    \section{Imputing age and gender}\label{imputing-age-and-gender}

    \subsection{(a)}\label{a}

Using the survey dataset, we can estimate models to predict gender and
age by weight and total income, and then plug in weight and (labor
income + capital income) in the BestIncome.txt dataset to impute gender
and age. Here we assume that total income = labor income + capital
income. We use log-linear model to predict gender and linear model to
predict age. Following are our model equations:

\(log(\frac{p(female_i)}{1-p(female_i)}) = \alpha_1 + \beta_{11} tot\_inc_i + \beta_{12} wgt_i\)

\(age_i = \alpha_2 + \beta_{21} tot\_inc_i + \beta_{22} wgt_i\)

Where \(tot\_inc_i = lab\_inc_i+cap\_inc_i\)

\subsection{(b)}\label{b}

    \begin{Verbatim}[commandchars=\\\{\}]
{\color{incolor}In [{\color{incolor}1}]:} \PY{k+kn}{import} \PY{n+nn}{pandas} \PY{k}{as} \PY{n+nn}{pd}
        \PY{k+kn}{import} \PY{n+nn}{numpy} \PY{k}{as} \PY{n+nn}{np}
        \PY{k+kn}{import} \PY{n+nn}{statsmodels}\PY{n+nn}{.}\PY{n+nn}{api} \PY{k}{as} \PY{n+nn}{sm}
        \PY{n}{BestIncome} \PY{o}{=} \PY{n}{pd}\PY{o}{.}\PY{n}{read\PYZus{}csv}\PY{p}{(}\PY{l+s+s1}{\PYZsq{}}\PY{l+s+s1}{BestIncome.txt}\PY{l+s+s1}{\PYZsq{}}\PY{p}{,} \PY{n}{header}\PY{o}{=}\PY{k+kc}{None}\PY{p}{)}
        \PY{n}{SurvIncome} \PY{o}{=} \PY{n}{pd}\PY{o}{.}\PY{n}{read\PYZus{}csv}\PY{p}{(}\PY{l+s+s1}{\PYZsq{}}\PY{l+s+s1}{SurvIncome.txt}\PY{l+s+s1}{\PYZsq{}}\PY{p}{,} \PY{n}{header}\PY{o}{=}\PY{k+kc}{None}\PY{p}{)}
        \PY{n}{Best\PYZus{}ary} \PY{o}{=} \PY{n}{BestIncome}\PY{o}{.}\PY{n}{values}
        \PY{n}{Surv\PYZus{}ary} \PY{o}{=} \PY{n}{SurvIncome}\PY{o}{.}\PY{n}{values}
        \PY{n}{Surv\PYZus{}female} \PY{o}{=} \PY{n}{Surv\PYZus{}ary}\PY{p}{[}\PY{p}{:}\PY{p}{,}\PY{l+m+mi}{3}\PY{p}{]}
        \PY{n}{Surv\PYZus{}age} \PY{o}{=} \PY{n}{Surv\PYZus{}ary}\PY{p}{[}\PY{p}{:}\PY{p}{,}\PY{l+m+mi}{2}\PY{p}{]}
        \PY{n}{Surv\PYZus{}x} \PY{o}{=} \PY{n}{Surv\PYZus{}ary}\PY{p}{[}\PY{p}{:}\PY{p}{,}\PY{p}{:}\PY{l+m+mi}{2}\PY{p}{]}
        \PY{n}{Surv\PYZus{}x} \PY{o}{=} \PY{n}{sm}\PY{o}{.}\PY{n}{add\PYZus{}constant}\PY{p}{(}\PY{n}{Surv\PYZus{}x}\PY{p}{)}
        
        \PY{n}{model\PYZus{}female} \PY{o}{=} \PY{n}{sm}\PY{o}{.}\PY{n}{Logit}\PY{p}{(}\PY{n}{Surv\PYZus{}female}\PY{p}{,}\PY{n}{Surv\PYZus{}x}\PY{p}{)}\PY{o}{.}\PY{n}{fit}\PY{p}{(}\PY{p}{)}
        \PY{n}{model\PYZus{}age} \PY{o}{=} \PY{n}{sm}\PY{o}{.}\PY{n}{OLS}\PY{p}{(}\PY{n}{Surv\PYZus{}age}\PY{p}{,}\PY{n}{Surv\PYZus{}x}\PY{p}{)}\PY{o}{.}\PY{n}{fit}\PY{p}{(}\PY{p}{)}
        
        \PY{n}{Best\PYZus{}x} \PY{o}{=} \PY{n}{np}\PY{o}{.}\PY{n}{transpose}\PY{p}{(}\PY{p}{[}\PY{n}{Best\PYZus{}ary}\PY{p}{[}\PY{p}{:}\PY{p}{,}\PY{l+m+mi}{0}\PY{p}{]}\PY{o}{+}\PY{n}{Best\PYZus{}ary}\PY{p}{[}\PY{p}{:}\PY{p}{,}\PY{l+m+mi}{1}\PY{p}{]}\PY{p}{,}\PY{n}{Best\PYZus{}ary}\PY{p}{[}\PY{p}{:}\PY{p}{,}\PY{l+m+mi}{3}\PY{p}{]}\PY{p}{]}\PY{p}{)}
        \PY{n}{Best\PYZus{}x} \PY{o}{=} \PY{n}{sm}\PY{o}{.}\PY{n}{add\PYZus{}constant}\PY{p}{(}\PY{n}{Best\PYZus{}x}\PY{p}{)}
        \PY{n}{Best\PYZus{}female} \PY{o}{=} \PY{n}{model\PYZus{}female}\PY{o}{.}\PY{n}{predict}\PY{p}{(}\PY{n}{Best\PYZus{}x}\PY{p}{)}
        \PY{n}{Best\PYZus{}female} \PY{o}{=} \PY{n}{np}\PY{o}{.}\PY{n}{array}\PY{p}{(}\PY{n+nb}{list}\PY{p}{(}\PY{n+nb}{map}\PY{p}{(}\PY{k}{lambda} \PY{n}{p}\PY{p}{:} \PY{l+m+mi}{0} \PY{k}{if} \PY{n}{p}\PY{o}{\PYZlt{}}\PY{l+m+mf}{0.5} \PY{k}{else} \PY{l+m+mi}{1}\PY{p}{,} \PY{n}{Best\PYZus{}female}\PY{p}{)}\PY{p}{)}\PY{p}{)}
        \PY{n}{Best\PYZus{}age} \PY{o}{=} \PY{n}{model\PYZus{}age}\PY{o}{.}\PY{n}{predict}\PY{p}{(}\PY{n}{Best\PYZus{}x}\PY{p}{)}
\end{Verbatim}


    \begin{Verbatim}[commandchars=\\\{\}]
Optimization terminated successfully.
         Current function value: 0.036050
         Iterations 11

    \end{Verbatim}

    \subsection{(c)}\label{c}

    \begin{Verbatim}[commandchars=\\\{\}]
{\color{incolor}In [{\color{incolor}2}]:} \PY{k}{def} \PY{n+nf}{descripstats}\PY{p}{(}\PY{n}{v}\PY{p}{)}\PY{p}{:}
            \PY{k}{return} \PY{p}{\PYZob{}}\PY{l+s+s1}{\PYZsq{}}\PY{l+s+s1}{mean}\PY{l+s+s1}{\PYZsq{}}\PY{p}{:}\PY{n}{np}\PY{o}{.}\PY{n}{mean}\PY{p}{(}\PY{n}{v}\PY{p}{)}\PY{p}{,} \PY{l+s+s1}{\PYZsq{}}\PY{l+s+s1}{std}\PY{l+s+s1}{\PYZsq{}}\PY{p}{:}\PY{n}{np}\PY{o}{.}\PY{n}{std}\PY{p}{(}\PY{n}{v}\PY{p}{)}\PY{p}{,} \PY{l+s+s1}{\PYZsq{}}\PY{l+s+s1}{min}\PY{l+s+s1}{\PYZsq{}}\PY{p}{:}\PY{n+nb}{min}\PY{p}{(}\PY{n}{v}\PY{p}{)}\PY{p}{,} \PY{l+s+s1}{\PYZsq{}}\PY{l+s+s1}{max}\PY{l+s+s1}{\PYZsq{}}\PY{p}{:}\PY{n+nb}{max}\PY{p}{(}\PY{n}{v}\PY{p}{)}\PY{p}{,} \PY{l+s+s1}{\PYZsq{}}\PY{l+s+s1}{nob}\PY{l+s+s1}{\PYZsq{}}\PY{p}{:}\PY{n}{v}\PY{o}{.}\PY{n}{shape}\PY{p}{[}\PY{l+m+mi}{0}\PY{p}{]}\PY{p}{\PYZcb{}}
        
        \PY{n}{descripstats}\PY{p}{(}\PY{n}{Best\PYZus{}age}\PY{p}{)}
\end{Verbatim}


\begin{Verbatim}[commandchars=\\\{\}]
{\color{outcolor}Out[{\color{outcolor}2}]:} \{'mean': 44.890828412990999,
         'std': 0.21913910572901438,
         'min': 43.976494892939144,
         'max': 45.703819001557932,
         'nob': 10000\}
\end{Verbatim}
            
    \begin{Verbatim}[commandchars=\\\{\}]
{\color{incolor}In [{\color{incolor}3}]:} \PY{n}{Best\PYZus{}female} \PY{o}{=} \PY{n}{np}\PY{o}{.}\PY{n}{array}\PY{p}{(}\PY{n}{Best\PYZus{}female}\PY{p}{)}
        \PY{n}{descripstats}\PY{p}{(}\PY{n}{Best\PYZus{}female}\PY{p}{)}
\end{Verbatim}


\begin{Verbatim}[commandchars=\\\{\}]
{\color{outcolor}Out[{\color{outcolor}3}]:} \{'mean': 0.4546, 'std': 0.49793457401550256, 'min': 0, 'max': 1, 'nob': 10000\}
\end{Verbatim}
            
    \subsection{(d)}\label{d}

    \begin{Verbatim}[commandchars=\\\{\}]
{\color{incolor}In [{\color{incolor}4}]:} \PY{n}{Best\PYZus{}new} \PY{o}{=} \PY{n}{np}\PY{o}{.}\PY{n}{concatenate}\PY{p}{(}\PY{p}{(}\PY{n}{Best\PYZus{}ary}\PY{p}{,} \PY{n}{np}\PY{o}{.}\PY{n}{transpose}\PY{p}{(}\PY{p}{[}\PY{n}{Best\PYZus{}age}\PY{p}{,} \PY{n}{Best\PYZus{}female}\PY{p}{]}\PY{p}{)}\PY{p}{)}\PY{p}{,}\PY{n}{axis}\PY{o}{=}\PY{l+m+mi}{1}\PY{p}{)}
\end{Verbatim}


    \begin{Verbatim}[commandchars=\\\{\}]
{\color{incolor}In [{\color{incolor}5}]:} \PY{n}{np}\PY{o}{.}\PY{n}{corrcoef}\PY{p}{(}\PY{n}{Best\PYZus{}new}\PY{p}{)}
\end{Verbatim}


\begin{Verbatim}[commandchars=\\\{\}]
{\color{outcolor}Out[{\color{outcolor}5}]:} array([[ 1.        ,  0.99907936,  0.99965113, {\ldots},  0.99633716,
                 0.99965026,  0.99348508],
               [ 0.99907936,  1.        ,  0.99986386, {\ldots},  0.99175163,
                 0.99986445,  0.98768175],
               [ 0.99965113,  0.99986386,  1.        , {\ldots},  0.99373098,
                 0.99999997,  0.99012846],
               {\ldots}, 
               [ 0.99633716,  0.99175163,  0.99373098, {\ldots},  1.        ,
                 0.99372741,  0.99959121],
               [ 0.99965026,  0.99986445,  0.99999997, {\ldots},  0.99372741,
                 1.        ,  0.99012401],
               [ 0.99348508,  0.98768175,  0.99012846, {\ldots},  0.99959121,
                 0.99012401,  1.        ]])
\end{Verbatim}
            
    \section{Stationary and data drfit}\label{stationary-and-data-drfit}

    \subsection{(a)}\label{a}

    \begin{Verbatim}[commandchars=\\\{\}]
{\color{incolor}In [{\color{incolor}6}]:} \PY{n}{IncomeIntel} \PY{o}{=} \PY{n}{pd}\PY{o}{.}\PY{n}{read\PYZus{}csv}\PY{p}{(}\PY{l+s+s1}{\PYZsq{}}\PY{l+s+s1}{IncomeIntel.txt}\PY{l+s+s1}{\PYZsq{}}\PY{p}{,} \PY{n}{header}\PY{o}{=}\PY{k+kc}{None}\PY{p}{)}
        \PY{n}{IncomeIntel\PYZus{}ary}  \PY{o}{=} \PY{n}{IncomeIntel}\PY{o}{.}\PY{n}{values}
        \PY{n}{gre\PYZus{}qnt} \PY{o}{=} \PY{n}{IncomeIntel\PYZus{}ary}\PY{p}{[}\PY{p}{:}\PY{p}{,}\PY{l+m+mi}{1}\PY{p}{]}
        \PY{n}{salary\PYZus{}p4} \PY{o}{=} \PY{n}{IncomeIntel\PYZus{}ary}\PY{p}{[}\PY{p}{:}\PY{p}{,}\PY{l+m+mi}{2}\PY{p}{]}
        \PY{n}{x} \PY{o}{=}  \PY{n}{sm}\PY{o}{.}\PY{n}{add\PYZus{}constant}\PY{p}{(}\PY{n}{gre\PYZus{}qnt}\PY{p}{)}
        
        \PY{n}{model\PYZus{}incomeintel} \PY{o}{=} \PY{n}{sm}\PY{o}{.}\PY{n}{OLS}\PY{p}{(}\PY{n}{salary\PYZus{}p4}\PY{p}{,} \PY{n}{x}\PY{p}{)}\PY{o}{.}\PY{n}{fit}\PY{p}{(}\PY{p}{)}
        
        \PY{n}{model\PYZus{}incomeintel}\PY{o}{.}\PY{n}{summary}\PY{p}{(}\PY{p}{)}
\end{Verbatim}


\begin{Verbatim}[commandchars=\\\{\}]
{\color{outcolor}Out[{\color{outcolor}6}]:} <class 'statsmodels.iolib.summary.Summary'>
        """
                                    OLS Regression Results                            
        ==============================================================================
        Dep. Variable:                      y   R-squared:                       0.002
        Model:                            OLS   Adj. R-squared:                  0.001
        Method:                 Least Squares   F-statistic:                     2.277
        Date:                Thu, 11 Oct 2018   Prob (F-statistic):              0.132
        Time:                        14:03:33   Log-Likelihood:                -10507.
        No. Observations:                1000   AIC:                         2.102e+04
        Df Residuals:                     998   BIC:                         2.103e+04
        Df Model:                           1                                         
        Covariance Type:            nonrobust                                         
        ==============================================================================
                         coef    std err          t      P>|t|      [0.025      0.975]
        ------------------------------------------------------------------------------
        const       5.902e+04    744.576     79.260      0.000    5.76e+04    6.05e+04
        x1             1.7423      1.154      1.509      0.132      -0.523       4.008
        ==============================================================================
        Omnibus:                        1.692   Durbin-Watson:                   2.028
        Prob(Omnibus):                  0.429   Jarque-Bera (JB):                1.655
        Skew:                           0.009   Prob(JB):                        0.437
        Kurtosis:                       3.198   Cond. No.                     1.71e+03
        ==============================================================================
        
        Warnings:
        [1] Standard Errors assume that the covariance matrix of the errors is correctly specified.
        [2] The condition number is large, 1.71e+03. This might indicate that there are
        strong multicollinearity or other numerical problems.
        """
\end{Verbatim}
            
    As can be seen from the summary chart, the estimated coefficients are
\(\beta_0=59020\) and \(\beta_1=1.7423\), the corresponding standard
errors are 744.576 and 1.154

\subsection{(b)}\label{b}

    \begin{Verbatim}[commandchars=\\\{\}]
{\color{incolor}In [{\color{incolor}7}]:} \PY{k+kn}{import} \PY{n+nn}{matplotlib}\PY{n+nn}{.}\PY{n+nn}{pyplot} \PY{k}{as} \PY{n+nn}{plt}
        \PY{n}{plt}\PY{o}{.}\PY{n}{scatter}\PY{p}{(}\PY{n}{IncomeIntel\PYZus{}ary}\PY{p}{[}\PY{p}{:}\PY{p}{,}\PY{l+m+mi}{0}\PY{p}{]}\PY{p}{,} \PY{n}{gre\PYZus{}qnt}\PY{p}{)}
\end{Verbatim}


\begin{Verbatim}[commandchars=\\\{\}]
{\color{outcolor}Out[{\color{outcolor}7}]:} <matplotlib.collections.PathCollection at 0x10d219b70>
\end{Verbatim}
            
    \begin{center}
    \adjustimage{max size={0.9\linewidth}{0.9\paperheight}}{output_14_1.png}
    \end{center}
    { \hspace*{\fill} \\}
    
    A drift in the GRE quantitative score scale happened in 2011, which will
affect the estimated coefficients and the statistical significance of
the regression results. A change in variable scale means that the
coefficients in the two time period are different. A solution to this
problem is to standardize the GRE quantitative score by z score within
each year. Instead of using the raw score, we use the relative score,
which is not affected by the change of scale.

    \begin{Verbatim}[commandchars=\\\{\}]
{\color{incolor}In [{\color{incolor}8}]:} \PY{n}{IncomeIntel}\PY{o}{.}\PY{n}{columns} \PY{o}{=} \PY{p}{[}\PY{l+s+s1}{\PYZsq{}}\PY{l+s+s1}{grad\PYZus{}year}\PY{l+s+s1}{\PYZsq{}}\PY{p}{,} \PY{l+s+s1}{\PYZsq{}}\PY{l+s+s1}{gre\PYZus{}qnt}\PY{l+s+s1}{\PYZsq{}}\PY{p}{,} \PY{l+s+s1}{\PYZsq{}}\PY{l+s+s1}{salary\PYZus{}p4}\PY{l+s+s1}{\PYZsq{}}\PY{p}{]}
\end{Verbatim}


    \begin{Verbatim}[commandchars=\\\{\}]
{\color{incolor}In [{\color{incolor}9}]:} \PY{n}{IncomeIntel}\PY{p}{[}\PY{l+s+s1}{\PYZsq{}}\PY{l+s+s1}{gre\PYZus{}qnt}\PY{l+s+s1}{\PYZsq{}}\PY{p}{]} \PY{o}{=} \PY{n}{IncomeIntel}\PY{o}{.}\PY{n}{groupby}\PY{p}{(}\PY{l+s+s1}{\PYZsq{}}\PY{l+s+s1}{grad\PYZus{}year}\PY{l+s+s1}{\PYZsq{}}\PY{p}{)}\PY{o}{.}\PY{n}{transform}\PY{p}{(}\PY{k}{lambda} \PY{n}{x}\PY{p}{:} \PY{p}{(}\PY{n}{x} \PY{o}{\PYZhy{}} \PY{n}{x}\PY{o}{.}\PY{n}{mean}\PY{p}{(}\PY{p}{)}\PY{p}{)} \PY{o}{/} \PY{n}{x}\PY{o}{.}\PY{n}{std}\PY{p}{(}\PY{p}{)}\PY{p}{)}\PY{p}{[}\PY{l+s+s1}{\PYZsq{}}\PY{l+s+s1}{gre\PYZus{}qnt}\PY{l+s+s1}{\PYZsq{}}\PY{p}{]}
        
        \PY{n}{gre\PYZus{}qnt\PYZus{}new} \PY{o}{=} \PY{n}{IncomeIntel}\PY{p}{[}\PY{l+s+s1}{\PYZsq{}}\PY{l+s+s1}{gre\PYZus{}qnt}\PY{l+s+s1}{\PYZsq{}}\PY{p}{]}\PY{o}{.}\PY{n}{values}
        
        \PY{n}{x\PYZus{}new} \PY{o}{=}  \PY{n}{sm}\PY{o}{.}\PY{n}{add\PYZus{}constant}\PY{p}{(}\PY{n}{gre\PYZus{}qnt\PYZus{}new}\PY{p}{)}
\end{Verbatim}


    \subsection{(c)}\label{c}

    \begin{Verbatim}[commandchars=\\\{\}]
{\color{incolor}In [{\color{incolor}10}]:} \PY{n}{plt}\PY{o}{.}\PY{n}{scatter}\PY{p}{(}\PY{n}{IncomeIntel\PYZus{}ary}\PY{p}{[}\PY{p}{:}\PY{p}{,}\PY{l+m+mi}{0}\PY{p}{]}\PY{p}{,} \PY{n}{salary\PYZus{}p4}\PY{p}{)}
\end{Verbatim}


\begin{Verbatim}[commandchars=\\\{\}]
{\color{outcolor}Out[{\color{outcolor}10}]:} <matplotlib.collections.PathCollection at 0x1c138d94a8>
\end{Verbatim}
            
    \begin{center}
    \adjustimage{max size={0.9\linewidth}{0.9\paperheight}}{output_19_1.png}
    \end{center}
    { \hspace*{\fill} \\}
    
    \begin{Verbatim}[commandchars=\\\{\}]
{\color{incolor}In [{\color{incolor}11}]:}  \PY{n}{IncomeIntel}\PY{o}{.}\PY{n}{groupby}\PY{p}{(}\PY{l+s+s1}{\PYZsq{}}\PY{l+s+s1}{grad\PYZus{}year}\PY{l+s+s1}{\PYZsq{}}\PY{p}{)}\PY{o}{.}\PY{n}{mean}\PY{p}{(}\PY{p}{)}
\end{Verbatim}


\begin{Verbatim}[commandchars=\\\{\}]
{\color{outcolor}Out[{\color{outcolor}11}]:}                 gre\_qnt     salary\_p4
         grad\_year                            
         2001.0    -2.275642e-15  60838.451338
         2002.0     2.390945e-15  59153.892438
         2003.0     1.047791e-14  61933.955229
         2004.0     9.005783e-15  60082.972426
         2005.0    -3.734387e-15  60446.862671
         2006.0     3.771875e-15  59879.706881
         2007.0     1.089172e-14  59527.961768
         2008.0     8.398765e-17  60204.841721
         2009.0     1.214036e-15  59972.927510
         2010.0    -8.756343e-15  60739.635733
         2011.0     5.998088e-15  59583.988081
         2012.0    -6.821383e-15  59695.533374
         2013.0     1.549638e-14  58654.483748
\end{Verbatim}
            
    From the scatter plot and the chart above, we can see that although very
slightly, the mean of the salary over years are changing. To make the
salary data stationary over time, we standardize the salary by the
following equation:

\(standardized(salary_{ti}) = salary_{ti} \frac{mean(salary_{2001})}{mean(salary_t)}\)

    \begin{Verbatim}[commandchars=\\\{\}]
{\color{incolor}In [{\color{incolor}14}]:} \PY{n}{mean2001} \PY{o}{=} \PY{n}{IncomeIntel}\PY{o}{.}\PY{n}{groupby}\PY{p}{(}\PY{l+s+s1}{\PYZsq{}}\PY{l+s+s1}{grad\PYZus{}year}\PY{l+s+s1}{\PYZsq{}}\PY{p}{)}\PY{o}{.}\PY{n}{mean}\PY{p}{(}\PY{p}{)}\PY{o}{.}\PY{n}{loc}\PY{p}{[}\PY{l+m+mf}{2001.0}\PY{p}{,} \PY{l+s+s1}{\PYZsq{}}\PY{l+s+s1}{salary\PYZus{}p4}\PY{l+s+s1}{\PYZsq{}}\PY{p}{]}
         
         \PY{n}{IncomeIntel}\PY{p}{[}\PY{l+s+s1}{\PYZsq{}}\PY{l+s+s1}{salary\PYZus{}p4}\PY{l+s+s1}{\PYZsq{}}\PY{p}{]} \PY{o}{=} \PY{n}{IncomeIntel}\PY{o}{.}\PY{n}{groupby}\PY{p}{(}\PY{l+s+s1}{\PYZsq{}}\PY{l+s+s1}{grad\PYZus{}year}\PY{l+s+s1}{\PYZsq{}}\PY{p}{)}\PY{o}{.}\PY{n}{transform}\PY{p}{(}\PY{k}{lambda} \PY{n}{x}\PY{p}{:} \PY{n}{x}\PY{o}{*}\PY{n}{mean2001}\PY{o}{/}\PY{n}{x}\PY{o}{.}\PY{n}{mean}\PY{p}{(}\PY{p}{)}\PY{p}{)}\PY{p}{[}\PY{l+s+s1}{\PYZsq{}}\PY{l+s+s1}{salary\PYZus{}p4}\PY{l+s+s1}{\PYZsq{}}\PY{p}{]} 
         \PY{n}{salary\PYZus{}p4\PYZus{}new} \PY{o}{=} \PY{n}{IncomeIntel}\PY{p}{[}\PY{l+s+s1}{\PYZsq{}}\PY{l+s+s1}{salary\PYZus{}p4}\PY{l+s+s1}{\PYZsq{}}\PY{p}{]}\PY{o}{.}\PY{n}{values}
\end{Verbatim}


    \subsection{(d)}\label{d}

    \begin{Verbatim}[commandchars=\\\{\}]
{\color{incolor}In [{\color{incolor}15}]:} \PY{n}{model\PYZus{}incomeintel\PYZus{}new} \PY{o}{=} \PY{n}{sm}\PY{o}{.}\PY{n}{OLS}\PY{p}{(}\PY{n}{salary\PYZus{}p4\PYZus{}new}\PY{p}{,} \PY{n}{x\PYZus{}new}\PY{p}{)}\PY{o}{.}\PY{n}{fit}\PY{p}{(}\PY{p}{)}
         
         \PY{n}{model\PYZus{}incomeintel\PYZus{}new}\PY{o}{.}\PY{n}{summary}\PY{p}{(}\PY{p}{)}
\end{Verbatim}


\begin{Verbatim}[commandchars=\\\{\}]
{\color{outcolor}Out[{\color{outcolor}15}]:} <class 'statsmodels.iolib.summary.Summary'>
         """
                                     OLS Regression Results                            
         ==============================================================================
         Dep. Variable:                      y   R-squared:                       0.001
         Model:                            OLS   Adj. R-squared:                 -0.000
         Method:                 Least Squares   F-statistic:                    0.7072
         Date:                Thu, 11 Oct 2018   Prob (F-statistic):              0.401
         Time:                        14:05:32   Log-Likelihood:                -10518.
         No. Observations:                1000   AIC:                         2.104e+04
         Df Residuals:                     998   BIC:                         2.105e+04
         Df Model:                           1                                         
         Covariance Type:            nonrobust                                         
         ==============================================================================
                          coef    std err          t      P>|t|      [0.025      0.975]
         ------------------------------------------------------------------------------
         const       6.084e+04    283.227    214.804      0.000    6.03e+04    6.14e+04
         x1           239.7432    285.086      0.841      0.401    -319.695     799.181
         ==============================================================================
         Omnibus:                        1.816   Durbin-Watson:                   2.042
         Prob(Omnibus):                  0.403   Jarque-Bera (JB):                1.797
         Skew:                           0.014   Prob(JB):                        0.407
         Kurtosis:                       3.206   Cond. No.                         1.01
         ==============================================================================
         
         Warnings:
         [1] Standard Errors assume that the covariance matrix of the errors is correctly specified.
         """
\end{Verbatim}
            
    As can be seen from the summary chart, the estimated coefficients are
\(\beta_0=60840\) and \(\beta_1=239.7432\), the corresponding standard
errors are 283.227 and 285.086. \(\beta_0\) and its std don't change
much but \(\beta_1\) and its std in the new regression are over 100
times of those in the original regression. That's because the
standardization we did to the GRE quantitative score decreased the scale
of this data by over 100 times, while the standardization we did to the
salary didn't change the overall scale of it. Since the coefficient
\(\beta_1\) is not statistically significant, the regression result
gives no evidence that higher intelligence is associated with higher
income.

    \section{Assessment of Koissinets and
Watts}\label{assessment-of-koissinets-and-watts}

    This paper mainly answers the following research question: What roles do
choice homophily and induced homophily play in the emergence of
homophily (the observed tendency of people associating with similar
people) through the process of individuals in a social network
seletively make or break ties with others?

To address this quesiton, the paper explored the data based on the
population of 30,396 undergraduate and graduate students, faculty, and
staff in a large U.S. university, consisting of 3 datasets:

\begin{enumerate}
\def\labelenumi{\arabic{enumi}.}
\item
  the logs of e-mail interactions within the university over one
  academic year,
\item
  a database of individual attributes (status, gender, age, department,
  number of years in the community, etc.),
\item
  records of course registration
\end{enumerate}

The number of observations is 30396 for personal characteristics,
organizational affiliations, and course-related variables. The total
number of email interaction is 7,156,162. And the time period of the
data is 270 days. In APPENDIX A, we can find a description and
definition of all variables.

During the data cleaning process, a potential problem was that the
authors included only email messages that were sent to a single
recipient, to ensure that the data represent interpersonal
communication. This cleared away about 18\% of all emails. But in a
university context, group emails actually contain a lot of interpersonal
communication. For example professors often email their research
assistant in a group rather than in person, so that people in the group
will know what each other is doing. This might be a unique type of
interpersonal communication and eliminating it can lead to loss of a
propable source of homophily.

The difficulty of matching the email data to the theoretical construct
of "social relationships" lies in the discrete and intermittent feature
of email exchanges. It's hard to decide the timing of the formation and
dissolution a social link based on a "bursty" time series of email
exchanges between two persons. To address this problem, the authors
defined the instantaneous strength of a dyad at time t based on the
number of email exchanges in time window \(\tau\) and sampling period
\(\delta\). By carefully choosing \(\tau\), they decide the maximum time
at which a past interaction is assumed to contribute to the current
strength of relationship. And \(\delta\) determines whether events
separated in time will be treated as sequential or as simultaneous with
one another.


    % Add a bibliography block to the postdoc
    
    
    
    \end{document}
